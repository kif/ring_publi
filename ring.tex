%------------------------------------------------------------------------------
% Template file for the submission of papers to IUCr journals in LaTeX2e
% using the iucr document class
% Copyright 1999-2013 International Union of Crystallography
% Version 1.6 (28 March 2013)
%------------------------------------------------------------------------------

\documentclass[preprint]{iucr}              % DO NOT DELETE THIS LINE

     %-------------------------------------------------------------------------
     % Information about journal to which submitted
     %-------------------------------------------------------------------------
     \journalcode{J}              % Indicate the journal to which submitted
                                  %   A - Acta Crystallographica Section A
                                  %   B - Acta Crystallographica Section B
                                  %   C - Acta Crystallographica Section C
                                  %   D - Acta Crystallographica Section D
                                  %   E - Acta Crystallographica Section E
                                  %   F - Acta Crystallographica Section F
                                  %   J - Journal of Applied Crystallography
                                  %   M - IUCrJ
                                  %   S - Journal of Synchrotron Radiation

\begin{document}                  % DO NOT DELETE THIS LINE

     %-------------------------------------------------------------------------
     % The introductory (header) part of the paper
     %-------------------------------------------------------------------------

     % The title of the paper. Use \shorttitle to indicate an abbreviated title
     % for use in running heads (you will need to uncomment it).

\title{Automatic Debye-Scherrer ring extraction using computer vision}
%\shorttitle{Short Title}

     % Authors' names and addresses. Use \cauthor for the main (contact) author.
     % Use \author for all other authors. Use \aff for authors' affiliations.
     % Use lower-case letters in square brackets to link authors to their
     % affiliations; if there is only one affiliation address, remove the [a].

\author[a]{Saadia}{Shahzad}%{saadia.shahzad@pucit.edu.pk}
\author[a]{Nazar}{Khan}%{nazarkhan@pucit.edu.pk}
\author[a]{Zubair}{Nawaz}%{znawaz@pucit.edu.pk}
\author[b]{Claudio}{Ferrero}%{claudio.ferrero@esrf.fr}
\cauthor[b]{J\'er\^ome}{Kieffer}{jerome.kieffer@esrf.fr}

\aff[a]{PUCIT, Allama Iqbal Campus, University of the Punjab, Lahore,  
\country{Pakistan}}
\aff[b]{The European Synchrotron, 71 avenue des Martyrs, 38000 Grenoble
\country{France}}

     % Use \shortauthor to indicate an abbreviated author list for use in
     % running heads (you will need to uncomment it).

%\shortauthor{Soape, Author and Doe}

     % Use \vita if required to give biographical details (for authors of
     % invited review papers only). Uncomment it.

%\vita{Author's biography}

     % Keywords (required for Journal of Synchrotron Radiation only)
     % Use the \keyword macro for each word or phrase, e.g. 
     % \keyword{X-ray diffraction}\keyword{muscle}

%\keyword{keyword}

     % PDB and NDB reference codes for structures referenced in the article and
     % deposited with the Protein Data Bank and Nucleic Acids Database (Acta
     % Crystallographica Section D). Repeat for each separate structure e.g
     % \PDBref[dethiobiotin synthetase]{1byi} \NDBref[d(G$_4$CGC$_4$)]{ad0002}

%\PDBref[optional name]{refcode}
%\NDBref[optional name]{refcode}

\maketitle                        % DO NOT DELETE THIS LINE

\begin{synopsis}
Fully automatic extraction of Debye-Scherrer rings allows automatic calibration
of a diffraction setup. 
\end{synopsis}

\begin{abstract}
High performance synchrotron -> many frames

Needs precise calibration of the setup nefore the start of the experiment

Error-prone human intevention 

We present here a computer vision assisted procedure reducing the risk. 
\end{abstract}


     %-------------------------------------------------------------------------
     % The main body of the paper
     %-------------------------------------------------------------------------
     % Now enter the text of the document in multiple \section's, \subsection's
     % and \subsubsection's as required.

\section{Introduction}

Todo

\section{Computer vision ring extraction}

Ring extraction is a computer vision which can be tacked by many way,
one of the most powerful method is the Generalizing the Hough Transform
but if requires a sub-space sampling at many dimension, hardly applicable in
the case of diffraction images \ldots 

D.H. Ballard, "Generalizing the Hough Transform to Detect Arbitrary Shapes", Pattern Recognition, Vol.13, No.2, p.111-122, 1981

\subsection{Initial image segmentation}

\subsection{Region mergin procedure}

Text text text text text text text text text text text text text text
text text text text text text text.

\section{Recalibration of the geometry}

\section{Conclusion}
The code has been included into the pyFAI \cite{pyFAI_2015} program \ldots 


\bibliographystyle{iucr}
\bibliography{biblio}


\appendix
\section{Appendix title}

Text text text text text text text text text text text text text text
text text text text text text text.

\subsection{Title}

Text text text text text text text text text text text text text text
text text text text text text text.

\subsubsection{Title}

Text text text text text text text text text text text text text text
text text text text text text text.


     %-------------------------------------------------------------------------
     % The back matter of the paper - acknowledgements and references
     %-------------------------------------------------------------------------

     % Acknowledgements come after the appendices

\ack{Acknowledgements}
The author would like to thank the LinkSCEEM European program which initiated 
this collaboration \ldots  

     %-------------------------------------------------------------------------
     % TABLES AND FIGURES SHOULD BE INSERTED AFTER THE MAIN BODY OF THE TEXT
     %-------------------------------------------------------------------------

     % Simple tables should use the tabular environment according to this
     % model

\begin{table}
\caption{Caption to table}
\begin{tabular}{llcr}      % Alignment for each cell: l=left, c=center, r=right
 HEADING    & FOR        & EACH       & COLUMN     \\
\hline
 entry      & entry      & entry      & entry      \\
 entry      & entry      & entry      & entry      \\
 entry      & entry      & entry      & entry      \\
\end{tabular}
\end{table}

     % Postscript figures can be included with multiple figure blocks

%\begin{figure}
%\caption{Caption describing figure.}
%\includegraphics{fig1.ps}
%\end{figure}


\end{document}                    % DO NOT DELETE THIS LINE
%%%%%%%%%%%%%%%%%%%%%%%%%%%%%%%%%%%%%%%%%%%%%%%%%%%%%%%%%%%%%%%%%%%%%%%%%%%%%%
